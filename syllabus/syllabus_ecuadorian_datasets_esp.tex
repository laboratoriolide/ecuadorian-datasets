% Options for packages loaded elsewhere
\PassOptionsToPackage{unicode}{hyperref}
\PassOptionsToPackage{hyphens}{url}
\PassOptionsToPackage{dvipsnames,svgnames,x11names}{xcolor}
%
\documentclass[
  letterpaper,
  DIV=11,
  numbers=noendperiod]{scrartcl}

\usepackage{amsmath,amssymb}
\usepackage{iftex}
\ifPDFTeX
  \usepackage[T1]{fontenc}
  \usepackage[utf8]{inputenc}
  \usepackage{textcomp} % provide euro and other symbols
\else % if luatex or xetex
  \usepackage{unicode-math}
  \defaultfontfeatures{Scale=MatchLowercase}
  \defaultfontfeatures[\rmfamily]{Ligatures=TeX,Scale=1}
\fi
\usepackage{lmodern}
\ifPDFTeX\else  
    % xetex/luatex font selection
\fi
% Use upquote if available, for straight quotes in verbatim environments
\IfFileExists{upquote.sty}{\usepackage{upquote}}{}
\IfFileExists{microtype.sty}{% use microtype if available
  \usepackage[]{microtype}
  \UseMicrotypeSet[protrusion]{basicmath} % disable protrusion for tt fonts
}{}
\makeatletter
\@ifundefined{KOMAClassName}{% if non-KOMA class
  \IfFileExists{parskip.sty}{%
    \usepackage{parskip}
  }{% else
    \setlength{\parindent}{0pt}
    \setlength{\parskip}{6pt plus 2pt minus 1pt}}
}{% if KOMA class
  \KOMAoptions{parskip=half}}
\makeatother
\usepackage{xcolor}
\setlength{\emergencystretch}{3em} % prevent overfull lines
\setcounter{secnumdepth}{5}
% Make \paragraph and \subparagraph free-standing
\ifx\paragraph\undefined\else
  \let\oldparagraph\paragraph
  \renewcommand{\paragraph}[1]{\oldparagraph{#1}\mbox{}}
\fi
\ifx\subparagraph\undefined\else
  \let\oldsubparagraph\subparagraph
  \renewcommand{\subparagraph}[1]{\oldsubparagraph{#1}\mbox{}}
\fi


\providecommand{\tightlist}{%
  \setlength{\itemsep}{0pt}\setlength{\parskip}{0pt}}\usepackage{longtable,booktabs,array}
\usepackage{calc} % for calculating minipage widths
% Correct order of tables after \paragraph or \subparagraph
\usepackage{etoolbox}
\makeatletter
\patchcmd\longtable{\par}{\if@noskipsec\mbox{}\fi\par}{}{}
\makeatother
% Allow footnotes in longtable head/foot
\IfFileExists{footnotehyper.sty}{\usepackage{footnotehyper}}{\usepackage{footnote}}
\makesavenoteenv{longtable}
\usepackage{graphicx}
\makeatletter
\def\maxwidth{\ifdim\Gin@nat@width>\linewidth\linewidth\else\Gin@nat@width\fi}
\def\maxheight{\ifdim\Gin@nat@height>\textheight\textheight\else\Gin@nat@height\fi}
\makeatother
% Scale images if necessary, so that they will not overflow the page
% margins by default, and it is still possible to overwrite the defaults
% using explicit options in \includegraphics[width, height, ...]{}
\setkeys{Gin}{width=\maxwidth,height=\maxheight,keepaspectratio}
% Set default figure placement to htbp
\makeatletter
\def\fps@figure{htbp}
\makeatother

\KOMAoption{captions}{tableheading}
\makeatletter
\@ifpackageloaded{caption}{}{\usepackage{caption}}
\AtBeginDocument{%
\ifdefined\contentsname
  \renewcommand*\contentsname{Table of contents}
\else
  \newcommand\contentsname{Table of contents}
\fi
\ifdefined\listfigurename
  \renewcommand*\listfigurename{List of Figures}
\else
  \newcommand\listfigurename{List of Figures}
\fi
\ifdefined\listtablename
  \renewcommand*\listtablename{List of Tables}
\else
  \newcommand\listtablename{List of Tables}
\fi
\ifdefined\figurename
  \renewcommand*\figurename{Figure}
\else
  \newcommand\figurename{Figure}
\fi
\ifdefined\tablename
  \renewcommand*\tablename{Table}
\else
  \newcommand\tablename{Table}
\fi
}
\@ifpackageloaded{float}{}{\usepackage{float}}
\floatstyle{ruled}
\@ifundefined{c@chapter}{\newfloat{codelisting}{h}{lop}}{\newfloat{codelisting}{h}{lop}[chapter]}
\floatname{codelisting}{Listing}
\newcommand*\listoflistings{\listof{codelisting}{List of Listings}}
\makeatother
\makeatletter
\makeatother
\makeatletter
\@ifpackageloaded{caption}{}{\usepackage{caption}}
\@ifpackageloaded{subcaption}{}{\usepackage{subcaption}}
\makeatother
\ifLuaTeX
  \usepackage{selnolig}  % disable illegal ligatures
\fi
\usepackage{bookmark}

\IfFileExists{xurl.sty}{\usepackage{xurl}}{} % add URL line breaks if available
\urlstyle{same} % disable monospaced font for URLs
\hypersetup{
  pdftitle={Bases de Datos Ecuatorianas},
  pdfauthor={Laboratorio de Investigación para el Desarrollo del Ecuador},
  colorlinks=true,
  linkcolor={blue},
  filecolor={Maroon},
  citecolor={Blue},
  urlcolor={Blue},
  pdfcreator={LaTeX via pandoc}}

\title{Bases de Datos Ecuatorianas}
\usepackage{etoolbox}
\makeatletter
\providecommand{\subtitle}[1]{% add subtitle to \maketitle
  \apptocmd{\@title}{\par {\large #1 \par}}{}{}
}
\makeatother
\subtitle{Una breve revisión del panorama ecuatoriano de datos de
investigación}
\author{Laboratorio de Investigación para el Desarrollo del Ecuador}
\date{}

\begin{document}
\maketitle

\section{Descripción del curso}\label{descripciuxf3n-del-curso}

Este módulo proporciona una visión general de los datos existentes para
Ecuador, con un enfoque en conjuntos de datos que pueden ser
aprovechados para la investigación en ciencias sociales cuantitativas. A
menudo, un investigador puede encontrar obstáculos encontrando datos que
respondan a sus preguntas de investigación. Este módulo tiene como
objetivo proporcionar las herramientas para encontrar y acceder a
conjuntos de datos existentes, así como entender los problemas
metodológicos que implican su uso. Se introducen los tipos de conjuntos
de datos existentes, dónde encontrarlos, las instituciones que los
producen, cómo acceder a ellos, los problemas metodológicos y el
potencial para investigaciones futuras. El módulo está diseñado para
quiénes estén interesados en realizar investigaciones enfocadas en el
Ecuador, pero que no están familiarizados con el panorama de datos
existente. \textbf{No} se contempla la recolección de datos mediante
experimentos o encuestas, sino el uso de datos recolectados por otras
personas o instituciones.

\section{Contenidos}\label{sec-contents}

El siguiente es un esquema planificado del curso. Esto puede cambiar
dependiendo del ritmo de la clase. Los materiales del módulo, incluidos
las diapositivas, enlaces, conjuntos de datos (si es que aplica) y todos
los demás archivos, se publicarán en el Google Drive del programa.

\subsection{Clase 1: Introducción y conjuntos de datos
básicos}\label{clase-1-introducciuxf3n-y-conjuntos-de-datos-buxe1sicos}

\begin{itemize}
\item
  Introducción al módulo
\item
  Tipos de conjuntos de datos y como se presentan ante el usuario
\item
  Panorama general de los datos ecuatorianos

  \begin{itemize}
  \tightlist
  \item
    Instituciones públicas (gobiernos central y locales)
  \item
    Iniciativas no gubernamentales (académicas y de la sociedad civil)
  \item
    Organizaciones internacionales
  \end{itemize}
\item
  Principales instituciones públicas que producen conjuntos de datos

  \begin{itemize}
  \tightlist
  \item
    Instituto Nacional de Estadística y Censos (INEC)
  \item
    Banco Central del Ecuador (BCE)
  \item
    Registro Civil
  \item
    Servicio de Rentas Internas (SRI)
  \item
    Instituto Ecuatoriano de Seguridad Social (IESS)
  \end{itemize}
\end{itemize}

\begin{itemize}
\tightlist
\item
  Principales conjuntos de datos de encuestas para investigadores

  \begin{itemize}
  \tightlist
  \item
    Encuestas de empleo - Encuesta Nacional de Empleo, Desempleo y
    Subempleo (ENEMDU)
  \item
    Censo de Población y Vivienda
  \item
    Encuestas de salud - Encuesta Nacional de Salud y Nutrición
    (ENSANUT)
  \item
    Otros
  \end{itemize}
\item
  Revisión a fondo: Encuesta Nacional de Empleo, Desempleo y Subempleo
  (ENEMDU)

  \begin{itemize}
  \tightlist
  \item
    ¿De qué trata el conjunto de datos?
  \item
    ¿Qué se puede hacer con él?
  \item
    ¿Cómo acceder a él?
  \item
    ¿Dónde encontrar la documentación?
  \end{itemize}
\item
  Principales conjuntos de datos administrativos para investigadores

  \begin{itemize}
  \tightlist
  \item
    Datos de empleo y salarios - Registro Estadístico de Empleo en la
    Seguridad Social (REESS)
  \item
    Datos empresariales - Directorio de Compañías (Superintendencia de
    Compañías) y Directorio de Empresas y Establecimientos (INEC)
  \item
    Datos tributarios: ventas, compras, impuesto al valor agregado,
    cierres y suspensiones (SRI)
  \item
    Voto y elecciones - Registro Electoral (Consejo Nacional Electoral)
  \end{itemize}
\item
  Sitios web agregadores útiles

  \begin{itemize}
  \tightlist
  \item
    Banco de Datos Abiertos Ecuador (Gobierno Central)
  \item
    Banco de Datos Abiertos INEC
  \item
    Archivo Nacional de Metadatos Estadísticos (INEC)
  \end{itemize}
\end{itemize}

\subsection{Clase 2: Más allá de lo
básico}\label{clase-2-muxe1s-alluxe1-de-lo-buxe1sico}

\begin{itemize}
\tightlist
\item
  Iniciativas de \emph{Gobierno Abierto}

  \begin{itemize}
  \tightlist
  \item
    Gobierno Abierto Quito, Quito Cómo Vamos.
  \item
    Banco de Datos Abiertos Ecuador (Gobierno Central)
  \end{itemize}
\item
  Conjuntos de datos internacionales a nivel de país

  \begin{itemize}
  \tightlist
  \item
    Banco Mundial (WDI, WGI)
  \item
    Banco Interamericano de Desarrollo
  \item
    Naciones Unidas
  \item
    CEPAL
  \end{itemize}
\item
  Encuestas internacionales

  \begin{itemize}
  \tightlist
  \item
    Barómetro de las Américas
  \item
    Latinobarómetro
  \item
    Encuesta Mundial de Valores
  \end{itemize}
\item
  Iniciativas internacionales y no gubernamentales:

  \begin{itemize}
  \tightlist
  \item
    Datalat
  \item
    Observatorio Legislativo
  \item
    Observatorio de Contratación Pública (Fundación Ciudadanía y
    Desarrollo)
  \item
    Asobanca
  \end{itemize}
\item
  Datos especiales

  \begin{itemize}
  \tightlist
  \item
    Geoportal INEC
  \item
    Datos macroeconómicos: BCE, CEPAL, FMI, Banco Mundial, Cordes
    (NowCast).
  \end{itemize}
\item
  Desafíos metodológicos en el uso de conjuntos de datos ecuatorianos

  \begin{itemize}
  \tightlist
  \item
    Metadatos y documentación
  \item
    Facilidad de acceso y disponibilidad
  \item
    Calidad de los datos, fiabilidad y consideraciones éticas
  \item
    Acceso a instituciones gubernamentales fuera de Ecuador
  \item
    Comparación con otros países de América Latina
  \item
    Limpieza y procesamiento de datos, problemas comunes
  \item
    Reproducibilidad en la investigación revisada por pares
  \item
    ¿Cómo utilizan los autores publicados los conjuntos de datos
    ecuatorianos?
  \end{itemize}
\end{itemize}

\section{Metodología}\label{metodologuxeda}

Las primeras dos clases se conducirán de forma sincrónica, con
demostraciones del instructor y discusiones en grupo. La participación
activa en la discusión y planteamiento de preguntas es recomendada para
sacar provecho a este módulo. En la tercera clase de ayudantía, se
planificarán sesiones personalizadas en donde los docentes pueden acudir
al instructor con preguntas específicas sobre sus proyectos de
investigación o una base de datos que le interese, y recibir
asesoramiento sobre cómo encontrar, entender y acceder a los datos
necesarios. De igual forma, es posible trabajar en temas más avanzados
como el análisis de datos, la limpieza y la visualización, o la citación
de datos en trabajos académicos. En caso de que el tiempo de las
primeras dos clases no sea suficiente para cubrir todos los temas, se
podrá utilizar la tercera clase para completar el contenido.



\end{document}
